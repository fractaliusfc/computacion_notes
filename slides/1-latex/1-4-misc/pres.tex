\documentclass[11pt]{beamer}
% \usetheme{Copenhagen}
\usetheme{PaloAlto}
\usepackage[utf8]{inputenc}
\usepackage[spanish]{babel}
\usepackage{amsmath}
\usepackage{amsfonts}
\usepackage{amssymb}
\usepackage{graphicx}
\graphicspath{{figures/}}
\usepackage{ragged2e}
\usepackage{listings}
\usepackage{xcolor}
\usepackage{hyperref}
\hypersetup{urlcolor=blue}
%\setbeamercovered{transparent} 
%\setbeamertemplate{navigation symbols}{} 
\setbeamertemplate{navigation symbols}{% 
\insertslidenavigationsymbol \insertframenavigationsymbol \insertsubsectionnavigationsymbol \insertsectionnavigationsymbol \insertdocnavigationsymbol \insertbackfindforwardnavigationsymbol \hspace{1em}%
\usebeamerfont{footline} 	\raisebox{1.2pt}[0pt][0pt]{\insertframenumber/\inserttotalframenumber}%
}
\logo{\includegraphics[width=1.4cm]{fac-logo-w}} 
\institute{Facultad de Ciencias \\ Universidad Nacional Autónoma de México} 
\setbeamertemplate{caption}[numbered]

\definecolor{lightgrey}{rgb}{0.9,0.9,0.9}
\definecolor{darkgreen}{rgb}{0,0.6,0}

\lstset{language=[LaTeX]TeX,
texcsstyle=*\bf\color{blue},
numbers=none,
breaklines=true,
keywordstyle=\color{darkgreen},
literate=*{\{} {{\textcolor{darkgreen}\{}}{1}%
			 {\}} {{\textcolor{darkgreen}\}}}{1}%
			 {[} {{\textcolor{darkgreen}[}}{1}%
			 {]} {{\textcolor{darkgreen}]}}{1}%
			 {\$} {{\textcolor{darkgreen}\$}}{1},
commentstyle=\color{red},
frame=none,
tabsize=2,
backgroundcolor=\color{lightgrey}
}

\title{Introducci\'on a \LaTeX}
\subtitle{Cosas útiles}
\author{Carlos Espinosa}
\date{Agosto, 2022} 
% \subject{} 

\justifying
\begin{document}
	\begin{frame}{}
		\titlepage
	\end{frame}

	\begin{frame}{Índice}
		\tableofcontents
	\end{frame}

	\section{Comentarios iniciales}
	\begin{frame}[containsverbatim]{Comentarios iniciales}
		\begin{itemize}
			\justifying
			\item Existen muchas herramientas para usar en \LaTeX\ así como opciones. 
			Desafortunadamente, un curso completo de \LaTeX\ nos llevaría casi un semestre.
			\item En esta presentación veremos algunas cosas útiles que complementan 
			las presentaciones anteriores.
			\item Recordar que algunas características dependen del tipo de documento 
			que eligan hacer.
		\end{itemize}
\end{frame}
	\section{Características varias}
	\subsection{Nuevos párrafos}
	\begin{frame}[containsverbatim]{Nuevas l\'ineas y p\'arrafos}
		Existen distintas maneras para definir un nuevo párrafo o nueva línea:
		\begin{itemize}
			\justifying
			\item Basta con dejar una línea en blanco entre párrafos.
			\item El comando \textbf{\textbackslash newline}
			\item El comando \textbf{\textbackslash \textbackslash}
			\item El comando \textbf{\textbackslash par}
		\end{itemize}
\end{frame}
	\subsection{Secciones y subsecciones}
	\begin{frame}[containsverbatim]{Agregando secciones}
	\justifying	
			Podemos organizar nuestro texto por secciones y sub secciones (incluso un tercer nivel subsub secciones). Tambi\'en podemos crear cap\'itulo. Los comandos para las distintas secciones son:
			\begin{itemize}
				\item \textbackslash chapter{...}
				\item \textbackslash section{...}
				\item \textbackslash subsection{...}
				\item \textbackslash subsubsection{...}
			\end{itemize}
			Existe un comando que crear\'a de forma autom\'atica un \'indice basandos\'e en todas las divisiones que hayamos hecho a lo largo del texto.
			\begin{lstlisting}
\tableofcontents	
			\end{lstlisting}
	\end{frame}
	\subsection{Tablas}
	\begin{frame}[containsverbatim]{Creando tablas}
		\justifying
			La creaci\'on de tablas en \LaTeX\ es más sencilla de lo que parece.
			Para definir una tabla utilizaremos el \emph{entorno} \textit{tabular}.
			Por ejemplo: 
			\begin{lstlisting}[basicstyle=\small]
\begin{center}
\begin{tabular}{ c c c }
 cell1 & cell2 & cell3 \\ 
 cell4 & cell5 & cell6 \\  
 cell7 & cell8 & cell9    
\end{tabular}
\end{center}			\end{lstlisting}

Nótese que aquí hemos definido el numero de columnas y la alineación del texto en ellas
desde el inicio. El número de filas se define de acuerdo al contenido de la tabla.
\end{frame}	

\begin{frame}[containsverbatim]{A\~nadiendo bordes}
			Si queremos a\~nadir bordes, el comando se transforma en:
			\begin{lstlisting}[basicstyle=\small]
\begin{center}
\begin{tabular}{ |c|c|c| } 
 \hline
 cell1 & cell2 & cell3 \\ 
 cell4 & cell5 & cell6 \\ 
 cell7 & cell8 & cell9
 \hline
\end{tabular}
\end{center}
			\end{lstlisting}
Nótese que las líneas que separan a las columnas se definen desde el inicio de la 
tabla. Las líneas que separan las filas se hace manualmente con el comando:
\begin{lstlisting}
\hline
\end{lstlisting}
\end{frame}
\begin{frame}[containsverbatim]{Agregando una descripción y referencias}
			Al igual que con las imágenes, y realmente con cualquier objeto de \LaTeX, podemos agregarle un t\'itulo y una etiqueta para referenciar una tabla:
			\begin{lstlisting}[basicstyle=\tiny]
Tabla \ref{tabla:datos} es un ejemplo de como referencias elementos de \LaTeX.
 
\begin{table}[h!]
\centering
\begin{tabular}{||c c c c||} 
 \hline
 Col1 & Col2 & Col2 & Col3 \\ [0.5ex] 
 \hline\hline
 1 & 6 & 87837 & 787 \\ 
 2 & 7 & 78 & 5415 \\
 3 & 545 & 778 & 7507 \\
 4 & 545 & 18744 & 7560 \\
 5 & 88 & 788 & 6344 \\ [1ex] 
 \hline
\end{tabular}
\caption{Tabla para probar t\'itulos y referencias}
\label{table:datos}
\end{table}
			\end{lstlisting}
\end{frame}
\subsection{Bibliografía}
\begin{frame}[containsverbatim]{Creando una bibliografía}
	\justifying
	\LaTeX\ provee herramientas básicas para el uso de una bibliografía en 
	nuestros documentos. Esta herramienta se llama Bib\TeX. Aunque existen paquetes 
	especializados que nos permitirán usar Bib\TeX\ con algunas mejoras.

	Una bibliografía en \LaTeX\ se basa en entrada bibliográficas. Para esto, podemos 
	agregarlas directamente al documento.
	
	\begin{lstlisting}
\begin{thebibliography}{4}
\bibitem{texbook}
Donald E. Knuth (1986) \emph{The \TeX{} Book}, Addison-Wesley Professional.

\bibitem{lamport94}
Leslie Lamport (1994) \emph{\LaTeX: a document preparation system}, Addison
Wesley, Massachusetts, 2nd ed.
\end{thebibliography}		
	\end{lstlisting}
\end{frame}
\begin{frame}[containsverbatim]{Creando una bibliografía}
	\justifying
	\begin{lstlisting}[basicstyle=\small]
\documentclass{article}
\usepackage[utf8]{inputenc}
\usepackage[spanish]{babel}
\usepackage{natbib}
\begin{document}
Aqui agregamos una referencia \cite{lamport94} y aqui agregamos la otra \cite{texbook}.
\begin{thebibliography}{4}
\bibitem{texbook}
Donald E. Knuth (1986) \emph{The \TeX{} Book}, Addison-Wesley Professional.
\bibitem{lamport94}
Leslie Lamport (1994) \emph{\LaTeX: a document preparation system}, Addison
Wesley, Massachusetts, 2nd ed.
\end{thebibliography}

\end{document}	
	\end{lstlisting}
\end{frame}
\begin{frame}[containsverbatim]{Creando una bibliografía}
	\justifying

	Agregar en el documento las entradas bibliográficas tiene ciertas desventajas,
	por ejemplo, si se require reutilizar la bibliografía con otros estilos, 
	o en otro orden o en otros documentos tendríamos que modificar las entradas.

	Para evitar estos inconvenientes se prefiere usar los archivos de datos de 
	entradas bibliográficas (con terminación .bib).

	En este archivo se encuentran todas las entradas bibliográficas que podríamos 
	necesitar. Se guardan cada uno de los datos con su propio identificador lo que
	que hace que podamos cambiar de estilo fácilmente.
	Se debe de tomar en cuenta que el orden de compilación debe de ser: pdflatex, bibtex, pdflatex, pdflatex.
	La bibliografía se debe de agregar con el comando \textbf{\textbackslash bibliography\{name\}}
\end{frame}
\section{Comentarios finales}
\begin{frame}[containsverbatim]{Comentarios Finales}
	\justifying
	\begin{itemize}
		\item \LaTeX\ es una poderosa herramienta con la cual se pueden crear toda clase de 
		\item Los paquetes nos presentan muchas opciones para cualquier cosa que necesitemos.
		\item Su uso puede requerir una curva de aprendizaje diferente a la de otros procesadores de texto.
		\item Una vez que se tiene cierta experiencia, su uso puede resultar bastante cómodo.
	\end{itemize}
\end{frame}
\end{document}