\documentclass[11pt]{beamer}
% \usetheme{Copenhagen}
\usetheme{PaloAlto}
\usepackage[utf8]{inputenc}
\usepackage[spanish]{babel}
\usepackage{amsmath}
\usepackage{amsfonts}
\usepackage{amssymb}
\usepackage{graphicx}
\graphicspath{{figures/}}
\usepackage{ragged2e}
\usepackage{listings}
\usepackage{xcolor}
\usepackage{hyperref}
\hypersetup{urlcolor=blue}
%\setbeamercovered{transparent} 
%\setbeamertemplate{navigation symbols}{} 
\setbeamertemplate{navigation symbols}{% 
\insertslidenavigationsymbol \insertframenavigationsymbol \insertsubsectionnavigationsymbol \insertsectionnavigationsymbol \insertdocnavigationsymbol \insertbackfindforwardnavigationsymbol \hspace{1em}%
\usebeamerfont{footline} 	\raisebox{1.2pt}[0pt][0pt]{\insertframenumber/\inserttotalframenumber}%
}
\logo{\includegraphics[width=1.4cm]{fac-logo-w}} 
\institute{Facultad de Ciencias \\ Universidad Nacional Autónoma de México} 
\setbeamertemplate{caption}[numbered]

\definecolor{lightgrey}{rgb}{0.9,0.9,0.9}
\definecolor{darkgreen}{rgb}{0,0.6,0}

\lstset{language=[LaTeX]TeX,
texcsstyle=*\bf\color{blue},
numbers=none,
breaklines=true,
keywordstyle=\color{darkgreen},
literate=*{\{} {{\textcolor{darkgreen}\{}}{1}%
			 {\}} {{\textcolor{darkgreen}\}}}{1}%
			 {[} {{\textcolor{darkgreen}[}}{1}%
			 {]} {{\textcolor{darkgreen}]}}{1}%
			 {\$} {{\textcolor{darkgreen}\$}}{1},
commentstyle=\color{red},
frame=none,
tabsize=2,
backgroundcolor=\color{lightgrey}
}

\title{Introducci\'on a \LaTeX}
\subtitle{Creando documentos}
\author{Carlos Espinosa}
\date{Agosto, 2022} 
% \subject{} 

\justifying
\begin{document}
	\begin{frame}
		\titlepage
	\end{frame}

	\begin{frame}
		\tableofcontents
	\end{frame}

	\section{Comentarios iniciales}
	\subsection{Archivos}
	\begin{frame}{¿Dondé/Comó se guardan los archivos?}
		\begin{itemize}
			\justifying
			\item Los archivos de \LaTeX\ son guardados en archivos de texto plano.
			\item Se deben de editar con editores de documento de texto plano, por ejemplo
			\textbf{Bloc de notas} o cualquier editor de código.
			\item Los editores online ya integran un editor de texto plano.
		\end{itemize}
	\end{frame}
	\subsection{Código}
	\begin{frame}{¿Comandos?}
		\begin{itemize}
			\item \LaTeX\ funciona a base de comandos/instrucciones. Algunos de estos
			son indispensables para la creación de un documento.
			\item Los comandos deben de escribirse tal cual se indican.
			\item \LaTeX\ hace un uso intensivo de algunos símbolos: $\{$, $\}$, $\backslash$, etc.
		\end{itemize}
		
		
	\end{frame}

	\section{Creando un documento}
	\subsection{Tipo de documento}	

		\begin{frame}[containsverbatim]{¿Comó se crea un documento?}
		 	\LaTeX\ funciona diferente a otros editores de texto. Lo primero que debe 
		 	de especificarse es el tipo de documento.
			\begin{lstlisting}
\documentclass{article}
			\end{lstlisting}
			Además debe de especificarse donde inicia y termina el documento.
			\begin{lstlisting}
\begin{document}
...
\end{document}
			\end{lstlisting}
			Por lo tanto el ejemplo completo es:
			\begin{lstlisting}
\documentclass{article}
\begin{document}
Hola Mundo
\end{document}
			\end{lstlisting}
	    \end{frame}	
	\begin{frame}{Ejemplos de tipos de documento}
	\justifying
				Un \emph{documentclass} es un estilo de documento predefinido que contiene el formato del documento deseado. Dependiendo del tipo de documento que se va a escribir se escoge la clase de documento.

			\scriptsize
			\begin{tabular}{|c|c|}
			\hline 
			Tipo & Descripción \\ 
			\hline 
			article & \begin{tabular}{@{}c@{}}Para artículos científicos, presentaciones, \\ reportes cortos, 
			
			documentaciones de programas, invitaciones, etc\end{tabular}
			  \\ 
			\hline 
			IEEEtran & Para artículos con el formato de la IEEE \\ 
			\hline 
			proc & Una clase para expedientes basada en la clase article \\ 
			\hline 
			report & \begin{tabular}{@{}c@{}}Para reportes largos que contienen varios capítulos, \\ libros pequeños, tesis,etc 
\end{tabular}
			 \\ 
			\hline 
			book & Para libros reales \\ 
			\hline 
			slides & Para diapositivas \\ 
			\hline 
			memoir & \begin{tabular}{@{}c@{}}Basado en la clase book, con ella se puede\\ crear cualquier tipo de documento \end{tabular}
			\\ 
			\hline 
			letter & Para escribir cartas \\ 
			\hline 
			beamer & Para escribir presentaciones \\ 
			\hline 
			\end{tabular} 
				
	\end{frame}
	\subsection{Opciones generales}
	\begin{frame}[containsverbatim]{Cambiando características del documento}
		Se puede cambiar el tamaño de la fuente agregando el valor deseado en el comando inicial.
		\begin{lstlisting}
\documentclass[12pt]{article}
		\end{lstlisting}
		También se puede cambiar el tipo de papel.
		\begin{lstlisting}
\documentclass[letterpaper]{article}
		\end{lstlisting}
		Pueden haber otras opciones útiles como: 
		\begin{lstlisting}
\documentclass[twoside]{article}
		\end{lstlisting}
	\end{frame}
	\subsection{Agregando un título}
	\begin{frame}[containsverbatim]{Agregando información para un título}
		\justifying
		Podemos definir el título del documento con la instrucción:
	    \begin{lstlisting}
\title{Primer documento}
		\end{lstlisting}
		Para agregar un autor usamos:
		\begin{lstlisting}
\author{Alan Smithee}
		\end{lstlisting}	
		Si se desea incluir una fecha, se utiliza:
		\begin{lstlisting}
\date{Agosto, 2022}
		\end{lstlisting}

		\LaTeX\ incluirá automaticamente los datos definidos al usar el comando
		\begin{lstlisting}
\maketitle
		\end{lstlisting}
	\end{frame}
	\begin{frame}[containsverbatim]{Agregando información para un título}
        \begin{lstlisting}
\documentclass{article}

\title{Primer documento}
\author{Alan Smithee}
\date{Agosto, 2022}

\begin{document}

\maketitle

Hola Mundo

\end{document}
		\end{lstlisting}
	\end{frame}

	\section{Comentarios finales}
	\begin{frame}[containsverbatim]{Cosas a recordar}
		\begin{itemize}
			\justifying
			\item Los comando en \LaTeX\ siempre inicial con una diagonal invertida: $\backslash$
			\item Los argumentos de los comando siempre van entre llaves: $\{\}$
			\item Las opciones de los comandos siempre van entre corchetes: $[]$
			\item Los comandos más importantes de un documento de \LaTeX\ son:
			\begin{itemize}
				\item \lstinline{\documentclass}
				\item \lstinline!\begin{document}!
				\item \lstinline!\end{document}!
			\end{itemize}
		\end{itemize}
	\end{frame}
\end{document}