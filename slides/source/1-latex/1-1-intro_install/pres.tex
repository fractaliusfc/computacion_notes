\documentclass[11pt]{beamer}
% \usetheme{Copenhagen}
\usetheme{PaloAlto}
\usepackage[utf8]{inputenc}
\usepackage[spanish]{babel}
\usepackage{amsmath}
\usepackage{amsfonts}
\usepackage{amssymb}
\usepackage{graphicx}
\graphicspath{{figures/}}
\usepackage{ragged2e}
\usepackage{listings}
\usepackage{xcolor}
\usepackage{hyperref}
\hypersetup{urlcolor=blue}
%\setbeamercovered{transparent} 
%\setbeamertemplate{navigation symbols}{} 
\logo{\includegraphics[width=1.4cm]{fac-logo-w}} 
\institute{Facultad de Ciencias \\ Universidad Nacional Autónoma de México} 
\setbeamertemplate{caption}[numbered]

\definecolor{lightgrey}{rgb}{0.9,0.9,0.9}
\definecolor{darkgreen}{rgb}{0,0.6,0}

\lstset{language=[LaTeX]TeX,
texcsstyle=*\bf\color{blue},
numbers=none,
breaklines=true,
keywordstyle=\color{darkgreen},
literate=*{\{} {{\textcolor{darkgreen}\{}}{1}%
			 {\}} {{\textcolor{darkgreen}\}}}{1}%
			 {[} {{\textcolor{darkgreen}[}}{1}%
			 {]} {{\textcolor{darkgreen}]}}{1}%
			 {\$} {{\textcolor{darkgreen}\$}}{1},
commentstyle=\color{red},
frame=none,
tabsize=2,
backgroundcolor=\color{lightgrey}
}

\title{Introducci\'on a \LaTeX}
\author{Carlos Espinosa}
\date{Agosto, 2022} 
% \subject{} 

\justifying
\begin{document}
	\begin{frame}
		\titlepage
	\end{frame}

	\begin{frame}
		\tableofcontents
	\end{frame}

	\section{Introducción}
	\subsection{¿Qu\'e es \LaTeX?}	

		\begin{frame}{¿Qu\'e es \LaTeX?}
			\begin{itemize}
			\justifying
			\item \LaTeX\ es un \textit{software} para la preparación de documentos 
			usado por académicos, investigadores, científicos, matemáticos, y 
			otros profesionales.

			\item A diferencia de otros editores (Microsoft Word, Google Docs, etc) que 
			se basan en el principio \textbf{WYSIWYG} (What You See Is What You Get), 
			\LaTeX\ se basa en funciones en documentos de texto plano que, junto a reglas 
			tipográficas, se compilan para generar documentos PDF.
			
			\item En el documento de texto plano se escribe el contenido, anotaciones y 
			comandos que controlan como se muestran diversos elementos. El resultado 
			es poder generar un documento de apariencia profesional.

			\item La creación de \textit{plantillas} en \LaTeX\ es sencillo, lo que 
			permite que cualquier persona pueda crear documentos complejos.
			\end{itemize}
		\end{frame}	
	\subsection{Historia de \LaTeX}
		\begin{frame}{Un poco de Historia}
			
			\begin{itemize}
			\justifying
			\item Donald Knuth inició el desarrollo de \TeX\ a finales de los 60s con el objetivo de 
			crear un programa que pudiera acomodar el texto y ecuaciones matemáticas 
			fácilmente. 
			\item \TeX\ nació en los 70s con un gran control y flexibilidad para la creación 
			de documentos. Sin embargo estas características hacían que \TeX\ fuera muy 
			complejo.
			\item A mediados de los 80s, Leslie Lamport introdujo características y \textit{macros} que hacían 
			mucho más sencillo el uso de \TeX. A partir de este momento se creo lo que conocemos como 
			\textit{Lamport-}\TeX\ o \LaTeX.
			\end{itemize}
	
		\end{frame}

	\subsection{Por qu\'e aprender \LaTeX?}	

		\begin{frame}{¿Por qu\'e aprender \LaTeX?}
			\begin{itemize}
				\item Es la mejor opción para la creación de cualquier documento 
				científico y técnico por sus herramientas especializadas para 
				generar documentos de alta calidad.
				\item \LaTeX\ provee una manera sencilla de escritura de ecuaciones matemáticas.
				\item No se tiene que preocupar por el formato del documento, solo del contenido.
				\item Plantillas que permiten que los autores no se tengan que preocupar por 
				cuestiones técnicas.
				\item Se pueden generar fácilmente referencias, índices, notas al pie de página y 
				citas.
				\item El documento es un documento de texto plano. 
				\item Se puede usar un software \textit{de control}.
				\item El documento final se puede generar en diversos formatos: PDF, DVI, PostScript, PNG, JPEG, etc.			\end{itemize}
		\end{frame}

	\section{Instalación}
	\subsection{¿Comó se obtiene \LaTeX?}

		\begin{frame}{¿Comó se obtiene \LaTeX?}
			\justifying

			\LaTeX\ es un \textbf{programa} que recibe una serie de \textit{comandos}
			que incluyen la configuración del documento y el contenido del mismo.

			Hay dos opciones para trabajar con \LaTeX, es decir, obtener el programa
			necesario tenemos dos opciones:
			\begin{itemize}
				\item Instalación local (Trabajando en una computadora personal)
				\item ``Instalación'' remota (Trabajando en un servicio online)
			\end{itemize}
		\end{frame}

	\subsubsection{Instalación Local}
		\begin{frame}{Instalación}
			\justifying
			
			Se puede instalar \LaTeX\ en los principales \textbf{sistemas operativos} actuales.
		    Existen diversos programas que nos proporcionan las herramientas de 
			\LaTeX\ necesarias para usarlo. Entre estas tenemos:

			\begin{itemize}
				\item \href{https://miktex.org/}{MiKTeX}
				\item \href{https://www.latex-project.org/}{The LaTeX Project}
				\item \href{https://www.tug.org/texlive/}{TeX Live}
				\item \href{https://www.tug.org/mactex/}{MacTeX} (MacOS)
			\end{itemize}


			Cada uno se instala como un programa. Se recomienda que se instale solo
			uno de ellos. 
		\end{frame}
		\begin{frame}{Instalación de editor}
			\justifying

			Los comandos de \LaTeX\ se guardan en archivos de texto plano con terminació
			\textbf{.tex}. Estos archivos son \textbf{compilados} por \LaTeX\ produciendo
			un archivo \textbf{PDF} como resultado final. Estos archivos se pueden
			editar con un programa sencillo, como el \textbf{bloc de notas} en Windows. Pero,
			es recomendable usar un editor especializado de \LaTeX.

			Existen diversos editores de \LaTeX, entre los cuales podemos encontrar:
			\begin{itemize}
				\item \href{https://www.xm1math.net/texmaker/}{TexMaker}
				\item \href{https://www.texstudio.org/}{TeXstudio}
			\end{itemize}

			Si se desea usar un editor de texto plano, podemos usar:
			\begin{itemize}
				\item \href{https://code.visualstudio.com/}{Visual Studio Code} (IDE)
				\item \href{https://www.sublimetext.com/}{Sublime Text}
				\item \href{https://brackets.io/}{Brackets}
				\item \href{https://notepad-plus-plus.org/}{Notepad++}
			\end{itemize}
		\end{frame}
	\subsubsection{Edición remota}
		\begin{frame}{Editores en Línea}
			\justifying

			Si no se desea instalar nada se puede optar por la opción en línea. Existen
			páginas que nos proporcionan todas las herramientas para generar un 
			documento de \LaTeX\ en línea. Entre estos editores podemos encontrar:
			
			\begin{itemize}
				\item \href{https://www.overleaf.com/}{Overleaf}
				\item \href{https://latexbase.com/}{\LaTeX\ Base}
				\item \href{https://papeeria.com/}{PapeeriA}
			\end{itemize}
			
			La ventaja de estas herramientas es no tener que instalar nada en las computadoras,
			lo único que necesitaremos es un explorador de internet. Además, algunos
			servicios prestan un servicio de almacenamiento con lo cual no nos tenemos 
			que preocupar por hacer respaldo de nuestros archivos.
		\end{frame}
		\begin{frame}
			\frametitle{Overleaf}
			
			\justifying

			El editor en línea recomendado es \href{https://www.overleaf.com/}{Overleaf}.
		    Tiene muchas características que son realmente útiles (aunque algunas de ellas
			son de pago). Además cuenta con una gran base de plantillas que nos permiten
			crear todo tipo de documentos.

			Se necesita crear una cuenta (la cual es gratis) para poder usarlo. Este será el
			editor que se usará a lo largo del curso.

			Pueden registrarse en esta liga: \url{https://bit.ly/3A5d0M1}

		\end{frame}
\end{document}
