\documentclass[11pt]{beamer}
% \usetheme{Copenhagen}
\usetheme{PaloAlto}
\usepackage[utf8]{inputenc}
\usepackage[spanish]{babel}
\usepackage{amsmath}
\usepackage{amsfonts}
\usepackage{amssymb}
\usepackage{graphicx}
\graphicspath{{figures/}}
\usepackage{ragged2e}
\usepackage{listings}
\usepackage{xcolor}
\usepackage{hyperref}
\hypersetup{urlcolor=blue}
%\setbeamercovered{transparent} 
%\setbeamertemplate{navigation symbols}{} 
\setbeamertemplate{navigation symbols}{% 
\insertslidenavigationsymbol \insertframenavigationsymbol \insertsubsectionnavigationsymbol \insertsectionnavigationsymbol \insertdocnavigationsymbol \insertbackfindforwardnavigationsymbol \hspace{1em}%
\usebeamerfont{footline} 	\raisebox{1.2pt}[0pt][0pt]{\insertframenumber/\inserttotalframenumber}%
}
\logo{\includegraphics[width=1.4cm]{fac-logo-w}} 
\institute{Facultad de Ciencias \\ Universidad Nacional Autónoma de México} 
\setbeamertemplate{caption}[numbered]

\definecolor{lightgrey}{rgb}{0.9,0.9,0.9}
\definecolor{darkgreen}{rgb}{0,0.6,0}

\lstset{language=[LaTeX]TeX,
texcsstyle=*\bf\color{blue},
numbers=none,
breaklines=true,
keywordstyle=\color{darkgreen},
literate=*{\{} {{\textcolor{darkgreen}\{}}{1}%
			 {\}} {{\textcolor{darkgreen}\}}}{1}%
			 {[} {{\textcolor{darkgreen}[}}{1}%
			 {]} {{\textcolor{darkgreen}]}}{1}%
			 {\$} {{\textcolor{darkgreen}\$}}{1},
commentstyle=\color{red},
frame=none,
tabsize=2,
backgroundcolor=\color{lightgrey}
}

\title{Introducci\'on a \LaTeX}
\subtitle{Usando paquetes}
\author{Carlos Espinosa}
\date{Agosto, 2022} 
% \subject{} 

\justifying
\begin{document}
	\begin{frame}{}
		\titlepage
	\end{frame}

	\begin{frame}{Índice}
		\tableofcontents
	\end{frame}

	\section{Comentarios iniciales}
	\subsection{Estilo de letra}
	\begin{frame}[containsverbatim]{Estilos de letras}
		\begin{itemize}
			\item Para poner las letras en \textbf{negritas} se utiliza el siguiente comando:
			
			\lstinline!\textbf{ejemplo}!
			
			\item Para poner las letras en \textit{cursiva} se utiliza el siguiente comando:
			
			\lstinline!\textit{ejemplo}!
			
			\item Para poner las letras \underline{subrayadas} se utiliza el siguiente comando:
			
			\lstinline!\underline{ejemplo}!
			
			\item Para poner las letras con \emph{énfasis} se utiliza el siguiente comando:
			
			\lstinline!\emph{ejemplo}!
			
			Este último depende del contexto dentro del que se use
		\end{itemize}
\end{frame}
	\subsection{Caracteres especiales}
		\begin{frame}[containsverbatim]{Caracteres especiales}
			Algunos caracteres se tienen que escribir de forma especial.
			
			Para obtener \# , escribe \lstinline!\#!
			
			Para obtener \$ , escribe \lstinline!\$!
			
			Para obtener \% , escribe \lstinline!\%!
			
			Para obtener \& , escribe \lstinline!\&!
			
			Para obtener \_ , escribe \lstinline!\_!
			
			Para obtener \{ o \} , escribe \lstinline!\{! o \lstinline!\}!
			
			Para obtener \~\ , escribe \lstinline!\~\!
			
			Para obtener \^\ , escribe \lstinline!\^\!
			
			Para obtener \textbackslash , escribe \lstinline!\textbackslash!
			
\end{frame}
			\begin{frame}[containsverbatim]{¿Y los acentos?}
				Dado que \LaTeX\ fue pensando inicialmente para escribir documentos 
				en inglés, no se tiene alguna instrucción para que tome en cuenta los acentos.

				Por lo tanto, para poder poner acentos, necesitamos escribir \lstinline!\'!:
				Por ejemplo:
				\begin{lstlisting}
coraz\'on
\'ultimo
\'enfasis
				\end{lstlisting}
				
				Esta es la forma internacional de poner acentos, aunque existen
				\emph{paquetes} que nos proporcionarán las herramientas para 
				escribir los acentos como estamos acostumbrados.
	\end{frame}

	\section{Paquetes}
		\begin{frame}[containsverbatim]{¿Qué es un paquete?}
			\begin{itemize}
				\item Un paquete es un conjunto de instrucciones que nos permiten 
				agregar funcionalidades a \LaTeX.
				\item Los paquetes son de uso libre y, generalmente, se puede encontrar 
				ayuda de la comunidad por si un error es encontrado.
				\item Cualquier persona puede crear un paquete y publicarlo.
				\item Es difícil decir cuantos paquetes existen actualmente dado 
				que no existe una base de paquetes que incluya todos.
				\item Para usar un paquete se debe de indicar el nombre, y las opciones
				correspondientes, al inicio del documento con la instrucción 
			\end{itemize}
			
			\begin{lstlisting}
\usepackage[opciones]{paquete}
			\end{lstlisting}
		\end{frame}
	\subsection{Utilidades para idiomas}	

		\begin{frame}[containsverbatim]{Para los acentos...}
			Aunque actualmente ya no se necesita un paquete para poner acentos, antes 
			se necesitaba del paquete \texttt{inputenc}.	

			Para usarlo se debe incluir la siguiente instrucción después de definir 
			la clase de documento:

			\begin{lstlisting}
\usepackage[utf8]{inputenc}
			\end{lstlisting}
			Por lo tanto el ejemplo completo es:
			\begin{lstlisting}
\documentclass{article}
\usepackage[utf8]{inputenc}
\begin{document}
...
\end{document}
			\end{lstlisting}
			Aunque ya no es necesario para poner acentos, puede ser útil para otros 
			símbolos que estén considerados dentro de la codificación \texttt{UTF-8}.
	    \end{frame}	
		\begin{frame}[containsverbatim]{¿\LaTeX\ no es amigable con otros idiomas?}
		Probemos el siguiente ejemplo:
			\begin{lstlisting}
\documentclass{article}
\usepackage[utf8]{inputenc}
\author{Carlos Espinosa}
\title{Primer Documento}
\date{\today}
\begin{document}
\maketitle
Hola mundo
\end{document}
			\end{lstlisting}
		Para que \LaTeX\ ponga en otro idioma todas las palabras predefinidas, usaremos 
		el paquete \texttt{babel} con la instrucción
			\begin{lstlisting}
\usepackage[spanish]{babel}
			\end{lstlisting}
		\end{frame}
	\subsection{Propiedades de la página}
		\begin{frame}[containsverbatim]{¿Comó puedo cambiar los márgenes de la página?}
			Cuando definimos el tipo de documento que queremos escribir, \LaTeX\ 
			carga los valores predeterminados para ese tipo de documento. Sin embargo,
			estos pueden ser muy altos, o bajos, para algunos usuarios.

			Existen varias maneras para cambiar los márgenes, otras propiedades de las 
			hojas, de nuestro documento. En este caso, la más sencilla es usar el paquete \texttt{geometry}.
			\begin{lstlisting}
\usepackage{geometry}
			\end{lstlisting}

		\end{frame}
		\begin{frame}[containsverbatim]{¿Comó puedo cambiar los márgenes de la página?}
			Posteriormente definiremos todas las opciones que necesitemos
			\begin{lstlisting}[basicstyle=\footnotesize]
\documentclass{article}
\usepackage{geometry}
\usepackage[utf8]{inputenc}
\geometry{letterpaper,top=2cm,left=2cm,right=2cm,bottom=2cm}
\author{Carlos Espinosa}
\title{Primer Documento}
\date{\today}
\begin{document}
\maketitle
% texto
\end{document}
			\end{lstlisting}			

		\end{frame}
	\subsection{Imágenes}
		\begin{frame}[containsverbatim]{Y las figuras?}
			\begin{itemize}
				\item Para incluir figuras/imagenes en un documento necesitamos del paquete \texttt{graphicx}.
			\item Nótese de la \textbf{x} al final. 
			\item Todas las imágenes deben de estar en la misma carpeta
			que el archivo \emph{.tex}. 
			\item El formato de las imágenes debe de estar en \textbf{PNG} o \textbf{JPG}.
			\item \LaTeX\ admite otro tipo de archivos pero para esto se deben de usar otras opciones.
			\end{itemize}
			\begin{lstlisting}
\usepackage{graphicx}
			\end{lstlisting}
			
		\end{frame}
		\begin{frame}[containsverbatim]{Y las figuras?}
			\begin{lstlisting}
\documentclass{article}
\usepackage{graphicx}
\author{Carlos Espinosa}
\title{Primer Documento}
\date{\today}
\begin{document}
\maketitle
Una region de formacion estelar, tambien conocida como region HII, es un gas ionizado
por estrellas jovenes y masivas

\includegraphics[scale=0.5]{orion.jpg}

Podemos ver a la nebulosa de Orion arriba.
\end{document}
			\end{lstlisting}			
		\end{frame}
		\begin{frame}[containsverbatim]{Y las figuras?}
			\begin{lstlisting}[basicstyle=\footnotesize]
\documentclass{article}
\usepackage{grpahicx}/viewer.html
\author{Carlos Espinosa}
\title{Primer Documento}
\date{\today}
\begin{document}
\maketitle
\begin{figure}
	\centering
	\includegraphics[scale=0.25\textwidth]{orion.jpg}
	\caption{Nebulosa de Orion}
	\label{fig:neb}
\end{figure}
En la figura \ref{fig:neb} podemos ver a la nebulosa de orion. En la p\'agina \pageref{fig:neb} se muestra una region HII.
\end{document}
			\end{lstlisting}			
		\end{frame}
	\subsection{Modo Matemático}
		\begin{frame}[containsverbatim]{Escribiendo ecuaciones matemáticas}
			\justifying
			Una de las mayores ventajas de usar \LaTeX\ es el modo matemático. 
			Uno de los mejores paquetes para escribir ecuaciones matemáticas es 
			\textbf{amsmath}. Este incluye todo lo necesario para escribir casi todas 
			las ecuaciones que se nos ocurran. Sin embargo, para complementar a este paquete 
			es buena idea que agreguemos el paquete \textbf{amsfonts}.

			\LaTeX\ proporciona dos \emph{modos} para escribir ecuaciones matemáticas:
			\begin{itemize}
				\justifying
				\item \textit{inline}: es el modo utilizado para escribir ecuaciones que son 
				parte de un párrafo.
				\item \textit{display}: es el modo utilizado para escribir ecuaciones que 
				no son parte de un párrafo, es decir, que es puesta en una línea nueva.
			\end{itemize}
		\end{frame}
		\begin{frame}[containsverbatim]{Modo matemático inline}
			\justifying
			Para escribir ecuaciones en modo inline, necesitamos poner nuestra expresión entre uno de los 
			siguientes delimitadores:
			\begin{itemize}
				\justifying
				\item \textbackslash(...\textbackslash) 
				\item \$...\$
				\item \textbackslash begin\{math\}...\textbackslash\{math\}
			\end{itemize}
			Por ejemplo:
			\begin{lstlisting}[basicstyle=\footnotesize]
\documentclass{article}
\usepackage{amsmath}
\usepackage{amsfonts}
\begin{document}
En f\'isica, la equivalencia entre masa y energ\'ia est\'a dada por la ecuaci\'on
$E=mc^2$, descubierta en 1905 por Albert Einstein.
\end{document}
			\end{lstlisting}			
		\end{frame}

		\begin{frame}[containsverbatim]{Modo matemático display}
			\justifying
			Hay dos maneras distintas en el modo display para mostrar las ecuaciones:
			\begin{itemize}
				\item Numeradas: \textbackslash begin\{equation\}...\textbackslash\{equation\}
				\item No numerada: \textbackslash[...\textbackslash]
			\end{itemize}
			\begin{lstlisting}[basicstyle=\footnotesize]
\documentclass{article}
\usepackage{amsmath}
\usepackage{amsfonts}
\begin{document}
En f\'isica, la equivalencia entre masa y energ\'ia est\'a dada por la ecuaci\'on
\[E=mc^2\]
descubierta en 1905 por Albert Einstein. En unidades naturales ($c=1$), la formula se 
expresa como:
\begin{equation}
	E=m
\end{equation}
\end{document}
			\end{lstlisting}			
		\end{frame}
    \section{Comentarios Finales}
		\begin{frame}[containsverbatim]{Comentarios finales}
			\begin{itemize}
			\justifying
			\item Los paquetes de \LaTeX\ nos permiten agregar funciones extra que nos 
			permiten crear documentos complejos.
		    
			\item Recordar que para usar un paquete se tiene que usar la instrucción: 
			\begin{lstlisting}
\usepackage[opciones]{paquete}
			\end{lstlisting}
			donde el nombre debe de ser exactamente el nombre del paquete.

			\item Cada paquete tiene sus propias opciones. Es bueno consultar el manual del paquete 
			en cuestión para saber como usarlo.
			\item Es bueno probar un paquete a la vez. Aunque es raro, a veces algunos paquetes 
			se contraponen.
			\item Tener en cuenta que cada paquete se tiene que cargar cuando se inicia la 
			compilación del archivo. Entre mas paquetes se usen, mas tardado será el tiempo
			de compilación.
			\end{itemize}
		\end{frame}
\end{document}